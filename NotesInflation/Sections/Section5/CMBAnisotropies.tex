\subsection{Introducing CMB temperature anisotropies}\label{sec:CMBAnisotropies}
To study how photons propagate between last scattering surface (LSS) and today consider a flat Universe subject to scalar perturbations (here $\Psi \neq -\Phi$ as we are not working in single perfect fluid approximation), change sign of $\Psi$ for convenience:
\begin{equation}
    ds^2 = a^2(\tau)\left[\left(1+2\Phi\right) - \left(1+2\Psi\right) \delta_{ij} dx^idx^j\right]
\end{equation}
Write down geodesic equation for photons, first with affine parameter $\lambda$, then in conformal time.
\begin{equation}
    \dot{P}^\beta + \Gamma^\beta_{\mu\nu}P^\mu P^\nu = 0 \qquad P^\mu= \frac{dx^\mu}{d \lambda} \qquad \dot{P}^\mu= \frac{dP^\mu}{d \lambda} = \frac{dP^\mu}{d \tau} P^0
\end{equation}
Look at $0$ component and then compute full perturbed connections at first order, in doing so expand $(1+2\phi)^{-1}$:
\begin{equation}
   \frac{dP^0}{d \tau} + \Gamma^0_{\mu\nu}\frac{P^\mu P^\nu}{P^0} = 0 
\end{equation}
Plug the expressions in and use the null condition $P^{\mu}P_{\nu}=0$ to get $P^i = P^0 \left(1+\Phi -\Psi \right) n^i$. 
Where $n^i = (1,1,1)$. This leads to 
\begin{equation}\label{eq:0thcomponentGeo}
    \frac{dP^0}{d \tau} + 2\frac{a'}{a}P = -P^0 \left(\Psi'+\Phi'\right) -2\Phi_{,i}n^i P^0
\end{equation}
Define the \textit{conformal momenta} \textcolor{darkgreen}{$\varTheta^\mu \equiv  a^2 P^\mu $} to notice that $\varTheta^0$ is conserved at zeroth order! This means that 
\emph{different photons of different frequency get red-shifted in the same way} $\,\Rightarrow\,$ the spectrum keeps its shape (at zeroth order).

Notice that the \emph{total} derivative of $\tau$ on $\Phi(\tau,x^i) = \Phi'+ \Phi_{,i}n^i$, so you can rewrite~\eqref{eq:0thcomponentGeo} as
\begin{equation}
    \frac{d\varTheta^0}{d \tau} = \varTheta^0 \left[\left(\Phi' - \Psi' \right) - 2  \frac{d\Phi}{d\tau}\right]
\end{equation}
This can be integrated expanding around $\varTheta^0(\tau'') \sim  \varTheta^0(\tau')$ which is motivated by the consideration above.
\begin{equation}
    \frac{\Theta^{0}(\tau'')-\Theta^{0}(\tau')}{\Theta^{0}(\tau')} = - 2\left[\Phi(\tau'')-\Phi(\tau')\right] + \int_{\tau'}^{\tau''} \left(\Phi'-\Psi'\right)d\tau 
\end{equation}

To get the relative frequency shift, consider first that an observer with four velocity $U^\mu$ measures $E=U_\mu P^\mu$
\begin{equation}
    E = aP^{0}\left(1+\Phi-\mathbf{n}\cdot\mathbf{\delta v}\right)
\end{equation}
So that, if we multiply both sides by $a(\tau)$ we construct $\varTheta$:
\begin{equation}
    \frac{\Theta^{0}(\tau'')-\Theta^{0}(\tau')}{\Theta^{0}(\tau')} = \frac{E a(\tau'')}{E a(\tau')}\Bigl[1-\Phi(\tau'')+\mathbf{n}\cdot \mathbf{v}(\tau'')+\Phi(\tau')-\mathbf{n} \cdot \mathbf{v}(\tau')\Bigr] - 1
\end{equation}
Now we wish to expand $E(\tau'')$, to do so first notice that $E(\tau'')a(\tau'')=E(\tau')a(\tau')$ at zeroth order. Hence $E(\tau'')a(\tau'') \sim \left[E(\tau')+ \Delta E(\tau'',\tau')\right]a(\tau')\,$. The relative frequency shift for a photon emitted at $\tau'$ and detected at $\tau''$ is thus
\begin{equation}
    \frac{\Delta E}{E}(\mathbf{n},\tau'',\tau') = \Phi(\tau')- \Phi(\tau'') +  \mathbf{n} \cdot \mathbf{v}(\tau') -\mathbf{n}\cdot \mathbf{v}(\tau'')+\Phi(\tau')  + \int_{\tau'}^{\tau''} \left(\Phi'-\Psi'\right)d\tau 
\end{equation}

Consider an ensemble of photon with temperature $T$ and set $\tau''=\tau_{now}$, $\tau'=\tau_{rec}$; then absorb the \textit{monopole term} $\Phi(\tau_{now})$ into the average temperature\footnote{CMB experiments only measure differences in temperature: $\Delta T(\mathbf{n})= T(\mathbf{n}) - T_{avg}$, thus, adding a constant to every photon’s energy (or to the mean temperature) simply shifts the zero-point of $T$.}
and neglect the \textit{dipole term} $\mathbf{n} \cdot \mathbf{v}(\tau_{now})$ as it accounts for the Doppler effect due to the relative motion of the observer with respect to the CMB rest frame\footnote{Tt comes from our local motion, not from fluctuations at recombination or along the line of sight; we subtract it in data processing: experiments fit and remove the dipole to isolate the cosmological signal.}.
\begin{eqopt}
    \frac{\delta T}{T}\left(\mathbf{n},\tau_{now}\right)
= \frac{\delta \rho_\gamma}{4\rho_\gamma}
   + \Phi(\tau_{\rm rec})+ n^{i}v_{i}(\tau_{\rm rec})
  + \int^{\tau_{now}}_{\tau_{\rm rec}}\!(\Phi'-\Psi')\,d\tau
\end{eqopt}
This is the \textit{Sachs-Wolfe equation}, which states that the observed temperature anisotropies originate from 
\begin{itemize}
\item Intrinsic density fluctuations at LSS 
\item Fluctuations of gravitational potential (gravitational red/blue shift) at LSS 
\item Doppler effects at LSS 
\item Contributions from time varying gravitational potential between LSS and detection
\end{itemize}