\subsection{CMB temperature anisotropies: Sachs-Wolfe plateau}\label{sec:CMBSWPlateau}
Consider perturbations that are superhorizon at recombination, $k\tau_{rec}<1$. Recall that in Section\ref{sec:SingCompFluid} we showed that $\delta \rho_m /\bar{\rho}_m = -2\Phi$ on superhorizon scales in a matter dominated Universe.
On scales much larger than the horizon, causally disconnected regions can’t “talk” to one another. Each patch evolves as if it were its own exact FRW cosmos, with its own scale factor, densities, pressures etc.
Instead of every patch having precisely the same cosmic time $t$ you say patch $\mathbf{x}$ has time $t+\delta t(\mathbf{x})$. \emph{The perturbation doesn’t come from heat flow or exchange with the outside, but from the fact that inflation left some regions a little “ahead” or “behind” in their evolution.}
\underline{Assume that the perturbation is adiabatic} $\,\Rightarrow\,$ all species share the same clock shift, therefore
\begin{equation}
    \frac{\delta \rho^i}{(\rho^i+p^i) } = -3H \delta t \quad \text{same }\forall i \quad \Rightarrow \quad \frac{\delta \rho_m}{(\rho_m+p_m) } =\frac{\delta \rho_\gamma}{(\rho_\gamma+p_\gamma) } =\frac{3}{4}\frac{\delta \rho_\gamma}{\rho_\gamma} 
\end{equation}
and thus, at recombination (when Universe is matter dominated), $\frac{\delta \rho_\gamma}{\rho_\gamma} = -\frac{8}{3}\Phi$.

For these modes, Sachs-Wolfe equation can be approximated by ignoring Doppler and time varying potential contribution:
\begin{equation}
    \frac{\Delta T_{now}}{T_{now}} (\mathbf{n}) \approx \frac{\delta \rho_\gamma}{4} (\tau_{rec}) + \Phi (\tau_{rec})= \frac{1}{3}\Phi
\end{equation}

Again from Section~\ref{sec:SingCompFluid}, recall that $\Phi_{\mathrm{matter}} = (9/10) \Phi_{\mathrm{rad}} \, \Rightarrow \, \Phi(\tau_{rec})=(9/10)\Phi(\tau_{infl})$, as $\Phi(\tau_{infl}) \equiv \Phi(\tau_{i})$ is constant on superhorizon scales.
\begin{center}
    \begin{tikzpicture}
  \begin{axis}[
    width=10cm, height=5cm,
    xlabel={$\tau$},
    ylabel={$\Phi(\tau)$},
    ymin=0, ymax=1.2,
    xmin=0, xmax=4, 
    xtick={1.5,3},
    xticklabels={$\tau_{\rm eq}$,$\tau_{\rm rec}$},
    ytick={0.7,1},
    yticklabels={$\tfrac{9}{10}\,\Phi(\tau_i)$,$\Phi(\tau_i)$},
    domain=0:4,
    axis lines=middle, 
    clip=false,
  ]
    % initial plateau at unity
    \addplot[black, thick] coordinates {(0,1) (1.5,1)};
    % step down
    \addplot[black, thick] coordinates {(1.5,1) (1.5,0.7)};
    % second plateau
    \addplot[black, thick] coordinates {(1.5,0.7) (4,0.7)};
  \end{axis}
\end{tikzpicture}
\end{center}
Therefore, 
\begin{equation}
     \frac{\Delta T_{now}}{T_{now}} (\mathbf{n}) \approx \frac{3}{10}\Phi(\tau_i) = \frac{3}{10} \int \frac{d^3k}{(2\pi)^{3/2}}\Phi(\mathbf{k},\tau_i) e^{i \mathbf{k}\mathbf{n}r_*}
\end{equation}
As before expand the plane wave in Legendre polinomials to reach
\begin{eqopt}[darkred]\label{eq:ClApprox}
    C_l = \frac{36 \pi}{100} \int \frac{dk}{k} \Delta_{\Phi}(k) j^2_l (k r_*)
\end{eqopt}
with \textcolor{darkgreen}{$\Delta_{\Phi}(k) \equiv \frac{k^3}{2 \pi^2} P_{\Phi}(k)$}. Then define \textcolor{darkgreen}{$\Delta_{\Phi}(k) \equiv \left(\frac{k}{k_*}\right)^{n_s-1} \mathcal{A}_{\Phi}$}. With \textcolor{darkred}{$n_s \approx 1$}~\eqref{eq:ClApprox} can be integrated to get 
\begin{eqopt}[darkred]
    C_l=\frac{18 \pi}{100} \frac{\mathcal{A}_\Phi}{l(l+1)}
\end{eqopt}
CMB data yelds
\begin{eqopt}[blue]
    \frac{l(l+1)C_l}{2\pi} \sim 10^{-10} \quad\textcolor{black}{\Rightarrow \quad \mathcal{A}_\Phi \approx 10^{-9}}
\end{eqopt}

We can use this bound to constrain the scale of inflation. Define the \textit{uniform density curvature perturbation} $\mathcal{\xi}$ as the curvature perturbation measured on hypersurfaces where the total energy density is unperturbed:
\begin{eqopt}[darkgreen]
   \mathcal{R} \equiv -\Phi - \frac{H}{\dot{\phi}}\delta \phi \qquad \mathcal{\xi} \equiv -\Phi + \frac{3(\delta \rho)}{\rho+p} 
\end{eqopt}
Notice that, on superhorizon scales, the comoving curvature perturbation $\mathcal{R}=u/z$ is $\approx\mathcal{\xi}= -\frac{3}{2}\Phi$ (in radiation dominated Universe as it is during inflation).  
We can also define the dimensionless power spectrum of $\mathcal{R}$ in the usual way: $\Delta_{\mathcal{R}} \equiv \frac{k^3}{2\pi^2} P_{\mathcal{R}_k} =\left(\frac{k}{k_*}\right)^{n_s-1} \mathcal{A}_{\mathcal{R}}$.
\begin{equation}
    \mathcal{R} \approx \mathcal{\xi}= -\frac{3}{2}\Phi \quad \Rightarrow \quad \Delta_{\mathcal{R}} = \frac{9}{4} \Delta_{\Phi} \quad \Rightarrow \quad \mathcal{A}_{\mathcal{R}} = \frac{9}{4}  \mathcal{A}_\Phi \approx 2\cdot 10^{-9}
\end{equation}
This implies $H\simeq 10^{-4} \sqrt{\epsilon} \,M_P$