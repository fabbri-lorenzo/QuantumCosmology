\subsection{Slow Roll Inflation}\label{sec:slowRollInfl}
We will assume that the field is homogeneous to zeroth order, consisting
of $\phi(t)+\delta\phi(t,\mathbf{x})$. Action for a scalar field $\phi$ with potential $V(\phi)$ in a FLRW universe 
\begin{eqopt}
    S = \int d^4x \sqrt{-g} \Bigg[ \frac{M_{P}^2}{2} R + \underbrace{\frac{1}{2} g^{\mu\nu} \partial_\mu \phi \, \partial_\nu \phi - V}_{\mathcal{L}} \Bigg]
\end{eqopt}
Knowing that, with mostly minus convention (then \underline{assume $k=0$}):
\begin{eqopt}[red]
R_{00}  =  \partial_0  \Gamma^k {}_{0 k}-\partial_k \Gamma^k {}_{0 0}  + \Gamma^\sigma {}_{0 k} \Gamma^k {}_{\sigma 0} - \Gamma^k {}_{k \sigma} \Gamma^\sigma {}_{0 0}   = 3 \frac{\ddot{a}}{a} \qquad R = 6 \left( \frac{\ddot{a}}{a} + \left( \frac{\dot{a}}{a} \right)^2 + \frac{k}{a^2} \right)
\end{eqopt}
Its equations of motion are given by the Einstein equations and the Klein-Gordon equation
\begin{equation}
   H^2 = \frac{1}{3M^2_P} \left(\frac{\dot{\phi}^2}{2} + V\right) \qquad \ddot{\phi} + 3 H \dot{\phi} = -V_{,\phi}  \label{eqMotionPhi}
\end{equation}
(From KG eq you can get $\dot{H} = -\dot{\phi}^2/2 M_{P}^2$). For a perfect fluid \textcolor{darkgreen}{$T^{\mu\nu} = (\rho + P) u^\mu u^\nu + P g^{\mu\nu}$},  
\begin{equation}
   \color{darkgreen}T^{\mu\nu} = \frac{2}{\sqrt{g}} \frac{\delta S_{\phi}}{\delta g_{\mu\nu}} \quad \color{black} \Rightarrow \quad  \omega =  \frac{\frac{\dot{\phi}^2}{2} - V(\phi)}{\frac{\dot{\phi}^2}{2} + V(\phi)}
\end{equation}
For $V(\phi)\gg \dot{\phi}^2/2 \rightarrow \omega \sim -1 \rightarrow \epsilon \ll 1$ and one has quasi de Sitter expansion. To quantify how long inflation lasts \footnote{$\epsilon$ should stay $\ll 1$ for a relevant physical time: \textcolor{red}{$\dot{\epsilon}\ll \epsilon H$}}, define 
\textcolor{darkgreen}{$\eta \equiv d\log{\epsilon}/dN$}$\quad\Rightarrow\quad \eta \ll 1$.

Rewrite~\eqref{eqMotionPhi} imposing slow roll approx for potential and field acceleration (vacuum energy dominates $H^2$ and terminal velocity). Then define
\begin{eqopt}[darkgreen]
    \epsilon_{V} \equiv \frac{M_{P}^2}{2} \left( \frac{V_{,\phi}}{V} \right)^2 \qquad \eta_{V} \equiv M_{P}^2 \frac{V_{,\phi\phi}}{V}
\end{eqopt}
\textcolor{darkred}{$ \epsilon = \epsilon_{V}$} and \textcolor{darkred}{$\eta = -2\eta_{V}+ 4\epsilon_{V}$}. Hence, $V$ must be flat enough to have $V_{,\phi}/V,\quad V_{,\phi\phi}/V\ll 1$.

For each model calculate $\epsilon_V, \eta_V$ and impose $\ll 1$ (for first model just $\epsilon_V$). Then find $\phi(N)$ to get the field value for 60 e-foldings. Rewrite $\epsilon_V(N), \eta_V(N)$

% Large Fields Models
\begin{mycolorbox}[pink!93!black]
    \textbf{Large Fields Models:}

    \begin{eqopt}
        V(\phi) = \lambda_n \phi^n \qquad n>0
    \end{eqopt}
    $\phi(N_{tot})\sim 15M_P$ and $\phi(start)\sim \sqrt{2}M_P$.
    \begin{equation}
       \epsilon_V = \frac{n}{4N} \qquad \eta_V = \frac{n-1}{2N}
    \end{equation}

    \begin{center}
        \begin{tikzpicture}[>=Stealth]
        \begin{axis}[
            axis lines=middle,
            xmin=-4.3,xmax=4.3,
            ymin=0,ymax=2.25,
            domain=-3.6:3.6,
            samples=200,
            axis line style={->,very thick},
            tick style={draw=none},
            xticklabels={}, yticklabels={},
            xlabel={$\,\phi$},
            ylabel={$V(\phi)$},
            xlabel style={below},
            ylabel style={left},
            clip=false,
            width=10cm,height=6cm
        ]
      
          %---- potential ---------------------------------------------------------
          \addplot[orange!80!black,thick, domain=-3.6:3.6] {0.12*x^2};
          % ---- tiny dashed continuations -----------------------
          \foreach \a/\b in {-3.95/-3.6, 3.6/3.95}{%
          \addplot[orange!80!black,thick,dashed, domain=-4:4] {0.12*x^2};
          }
      
         % -------- rolling ball + down arrow -------------------
        \addplot[only marks,
         mark=*,mark options={fill=gray!60!black},very thick,samples=1]
          coordinates {(3,0.13*3*3)};
        \draw[->,gray,thick] (axis cs:2.93,1.13) -- (axis cs:2.75,1);
      
          %---- vertical dashed cut‑offs -----------------------------------------
          \addplot[dashed] coordinates {(-2,-0.55) (-2,0.12*4)};
          \addplot[dashed] coordinates {( 2,-0.55) ( 2,0.12*4)};
          \node[below left] at (axis cs:-2,0) {$\phi_{\text{end}}$};
          \node[below right] at (axis cs: 2,0) {$\phi_{\text{end}}$};
      
          %---- inflation / no‑inflation regions ----------------------------------
          % left inflation 
          \node at (axis cs:-3.2,-0.45) {inflation};
          % middle -- no inflation
          \node at (axis cs:0,-0.45) {no inflation};
          % right inflation 
          \node at (axis cs:3.2,-0.45) {inflation};

      
        \end{axis}
      \end{tikzpicture}
    \end{center}
\end{mycolorbox}    

% Small Fields Hill-top & Inflection Models
\begin{mycolorbox}[pink!93!black]
    \textbf{Small Fields Hill-top \& Inflection Models:}

    \begin{eqopt}
        V(\phi) = V_0 - \lambda_n \phi^n \qquad n>0
    \end{eqopt}
    \begin{equation}
       \epsilon_V = \frac{n^2 \lambda^2}{2V_0^2} \left[n(n-2)N\frac{\lambda}{V_0}\right]^{\frac{2n-2}{2-n}} \qquad \eta_V =- \frac{n-1}{n-2}\frac{1}{N}
    \end{equation}

    \begin{center}
        \begin{tikzpicture}[>=Stealth]
            \begin{axis}[
                axis lines=middle,
                domain=-1.25:1.25,
                samples=400,
                xmin=-1.75, xmax=1.75,
                ymin=-1.5, ymax=2.75,          % keep the plateau in view
                clip=false,                   % let arrow heads stick out
                axis line style={->,very thick},
                tick style={draw=none},       % no tick marks
                xticklabels={}, yticklabels={},
                xlabel={$\,\phi$},
                ylabel={$V(\phi)$},
                xlabel style={below},
                ylabel style={left},
                width=12cm, height=6cm        % feel free to resize
            ]
              %--- the potentials ------------------------------------
              \addplot[very thick,orange!80!black]
                  {1 - x^2};
              \addplot[very thick,brown!80!black]
               {1 - 0.4*x^3};   
              % ---- tiny dashed continuations -----------------------
              \foreach \a/\b in {-1.45/-1.25, 1.25/1.45}{%
               \addplot[orange!80!black,thick,dashed, domain=\a:\b] {1 - x^2};
               }
                \foreach \a/\b in {-1.45/-1.25, 1.25/1.45}{%
              \addplot[brown!80!black,thick,dashed,domain=\a:\b]
                  {1 - 0.4*x^3};  } 
              \addlegendentry{Hill-top}
              \addlegendentry{Inflection}           
    \end{axis}
  \end{tikzpicture}
\end{center}
\end{mycolorbox}

% Plateau Models
\begin{mycolorbox}[pink!93!black]
    \textbf{Plateau Models:}

    \begin{eqopt}
        V(\phi) = V_0 \left( 1 - \alpha e^{-\kappa\phi} \right)^2 
    \end{eqopt}
 Neglect $\alpha^2$ terms in $V_\phi$ and $1 - \alpha e^{-\kappa\phi} \sim 1$ in $\epsilon_V$, $\eta_V$
     \begin{equation}
       \epsilon_V \sim \frac{1}{2k^2 N^2} \qquad \eta_V \sim - \frac{1}{N}
    \end{equation}

\begin{center}
            \begin{tikzpicture}[>=Stealth]
                \begin{axis}[
                    axis lines=middle,
                    domain=-0.5:4.5,
                    samples=400,
                    xmin=-1, xmax=5,
                    ymin=0, ymax=1.5,          % keep the plateau in view
                    clip=false,                   % let arrow heads stick out
                    axis line style={->,very thick},
                    tick style={draw=none},       % no tick marks
                    xticklabels={}, yticklabels={},
                    xlabel={$\,\phi$},
                    ylabel={$V(\phi)$},
                    xlabel style={below},
                    ylabel style={left},
                    width=12cm, height=6cm        % feel free to resize
                ]
                  %--- the potential ------------------------------------
                  \addplot[very thick,orange!80!black]
                      {(1 - exp(-x))^2};
                  % ---- tiny dashed continuation -----------------------
                  \addplot[orange!80!black,thick,dashed,domain=4.57:4.85]
                      {(1 - exp(-x))^2};   
                  \addplot[orange!80!black,thick,dashed,domain=-0.65:-0.5]
                      {(1 - exp(-x))^2};      
                  %--- plateau annotation -------------------------------
                  \node[above] at (axis cs:4,1) {plateau};
        \end{axis}
      \end{tikzpicture}
\end{center}
\end{mycolorbox}
