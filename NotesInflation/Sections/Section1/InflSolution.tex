\subsection{The actual problem: $aH$}
All comes down to $aH$ being a decreasing function of time. ($1/ah$ is the \textit{Hubble radius}.)

Use~\eqref{00}~\eqref{continuity} to relate proportionally $a$ and $t$. Then relate $aH$ and $t$, include multiplicative $omega$ factors!. Finally, relate  $aH$ and $a$. 
See that SEC ($\omega>-1/3$) must be violated to have $aH$ increasing. \textcolor{darkred}{$\tau \in \openleftinterval{0}{\infty}$} , violating SEC we can push the singularity to $\tau_i = -\infty$.

%Horizon Problem
\begin{mycolorbox}
    \textbf{Equivalent conditions to $\omega<-1/3$:}

    \begin{enumerate}
        \item Decreasing comoving horizon
        \item Accelerated expansion: $\ddot{a} >0$
        \item Slowly varying $H$ \hfill \textcolor{darkgreen}{$\epsilon_H \equiv -\frac{\dot{H}}{H^2}$}\textcolor{mypurple}{$=3/2 (1+\omega)$}
        \item Negative pressure
    \end{enumerate}
\end{mycolorbox}    

How much do we need to push back in time $\tau_i$ so that the whole observable universe could have been in causal contact and thus thermalize?
    
    