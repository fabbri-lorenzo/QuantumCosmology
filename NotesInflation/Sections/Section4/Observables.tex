\subsection{Observables from Inflation}\label{sec:ObsInfl}
In single‐field slow‐roll inflation the key “observables” are all evaluated at the moment each Fourier mode exits the Hubble radius during inflation. No need to know when inflation “begins” in some absolute sense. All that matters for observables is how many e-folds remain before inflation ends, because that determines when a given comoving scale $k$ crossed the Hubble radius.

In practice one picks a “pivot” wavenumber $k^* = a^* H$ (corresponding to $N^* \sim 50-60$) and defines $\mathcal{A}_s^*$, $n_s^*$, $r^*$, where $*$ means that are calculated at the pivot scale 
\begin{equation}
\log{\frac{a_{end}}{a(\phi)}}\equiv  N^* \sim \log{\frac{a_{end}H }{k^*}}
\end{equation}
Once a given mode has left the Hubble radius it “freezes in” (i.e.\ its curvature perturbation becomes conserved on superhorizon scales), so nothing in the microphysics of reheating or recombination changes the primordial values of $\mathcal{A}_s^*$, $n_s^*$, $r^*$ that were set at exit. 
\emph{What does happen between inflation and recombination is simply linear evolution through radiation and matter domination (via the transfer functions), but that evolution is completely determined by the background cosmology and does not alter the primordial spectral shape or amplitude.} Recall $\mathcal{A}_{s} \simeq 2 \cdot 10^{-9}$, $n_{s} \simeq 0.96$, $r \leq  7 \cdot 10^{-2}$.

As in Section~\ref{sec:slowRollInfl}, evaluate $\epsilon_V$, $\eta_V$ for each model. Then, get $H^* = 8 \pi^2 \epsilon A_s$ then $n_s$ and $r$.
% Large Fields Models
\begin{mycolorbox}[pink!93!black]
    \textbf{Large Fields/Monomial Models:}

    \begin{eqopt}[darkred]
       H^* \approx2.7\cdot 10^{-5} \sqrt{n}\,M_P \sim 10^5 GeV
    \end{eqopt}
    \begin{mycolorbox}[gray]
    \textbf{Duration of Inflation:}

    Approximating $H$ as constant during inflation, and using $H^* \approx2.7\cdot 10^{-5} \sqrt{n}\,M_P$
    \begin{equation}
        \Delta t = \frac{1}{H}\int_{a_{st}}^{a_{end}} \frac{d a}{a} \quad \Rightarrow \quad \Delta t \approx \frac{N}{2.7\cdot 10^{-5} \sqrt{n}M_P}\, \textcolor{darkred}{\sim 10^{-36} s}
    \end{equation}
        
    \end{mycolorbox}
\end{mycolorbox}    

% Small Fields Hill-top & Inflection Models
\begin{mycolorbox}[pink!93!black]
    \textbf{Small Fields/Hill-top \& Inflection Models:}

    Work in the regime $V_0 \gg \lambda_n \phi^n$, where you get $\eta_V \gg \epsilon_V$. Do the calculations for $n=4$.
    For slow roll models recall EOMs, in this case 
    \begin{equation}
        H^2 = \frac{1}{3M^2_P} \left(\frac{\dot{\phi}^2}{2} + V\right)  \sim \frac{V_0}{3}
    \end{equation}
    With this you can evaluate $A_s$ and the fix $\lambda$, you have only $V_0$ as free parameter.

\end{mycolorbox}

% Plateau Models
\begin{mycolorbox}[pink!93!black]
    \textbf{Plateau Models:}

   For $n_s$ discard $1/N^2$ term. For Starobinsky model ($\alpha=1, k=\sqrt{2/3}$) we have excellent agreement ($r \approx 0.004$).
\end{mycolorbox}