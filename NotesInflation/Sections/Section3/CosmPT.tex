\subsection{Cosmological Perturbation Theory}
\label{sec:cosmological_perturbation_theory}
The zeroth-order scheme outlined
in Section~\ref{sec:slowRollInfl} ensures that the universe will be close to uniform on all scales of
interest today. There are perturbations about this zeroth-order scheme, though, and these
perturbations—produced early on when the scales are causally connected—persist long
after inflation has terminated.

In cosmology, we always work in terms of statistics, such as the correlation function
and power spectrum, because no known theory predicts the overdensity in a given spot
on the sky. In the inflationary scenario, this uncertainty is fundamental: inflation erases all
traces of what came before it, and replaces those with quantum-mechanical vacuum fluctuations, 
which cannot be predicted in principle. What inflation predicts then is precisely
the statistical distributions from which the perturbations are drawn.

We are most interested in scalar perturbations to the metric since these couple to the
density of matter and radiation and ultimately are responsible for the structure we observe
in the universe. Inflation also generates tensor fluctuations in the metric, that is, gravitational
waves. These are not coupled to the density and so are not responsible for the large-scale structure 
of the universe, but they do induce anisotropies in the
CMB~\cite{ModernCosm}.

\begin{eqopt}\label{perturbation}
    g_{\mu\nu} = \bar{g}_{\mu\nu} + \delta g_{\mu\nu} \qquad T_{\mu\nu} = \bar{T}_{\mu\nu} + \delta T_{\mu\nu}
\end{eqopt}
Expanding the Einstein equations to first order in perturbations, we have:
\begin{equation}
    \delta G_{\mu\nu} = 8\pi G \, \delta T_{\mu\nu},
\end{equation}
\emph{Write down the most general linear perturbation that is compatible with the spatial symmetries of an FRW background, 
and then decompose every spatial object into irreducible pieces.} Background is homogeneous and isotropic $\rightarrow$ spatial rotations 
(plus translations) act as a symmetry group. Irreducible representations of $SO(3)$ are labelled by spin: 0 scalars, 1 vectors and 2 tensors. Use conformal time in the metric.
\begin{eqopt}\label{eq:fullMetric}
    \frac{ds^2}{a^2} = \left(1 + 2\Phi\right) d\tau^2 +2\left(B_{,i} + S_{i}\right)  d\tau dx^i - \left[(1 - 2\Psi)\delta_{ij} -2 E_{,ij} -F_{i,j}-F_{j,i}- h_{ij}\right ]  dx^i dx^j 
\end{eqopt} \vspace{-5mm}
\begin{eqopt}\label{eq:metricPT}
\delta g_{00} = 2a^2 \Phi \qquad \delta g_{0i} = a^2 \left(B_{,i} + S_i\right)\qquad \delta g_{ij} = a^2 \left(2\Psi \delta_{ij}+2E_{,ij}+F_{i,j}+F_{j,i}+ h_{ij}\right) 
\end{eqopt}
$S_i$, $F_i$ and $h_{ij}$ are transverse because they come from the decomposition of $V_i= \partial_i V + V^{\perp}_i$ and $W_{ij}=W\delta_{ij}+\partial_i\partial_j W+ \partial_i W_j + \partial_j W_i + W^{\perp}_{ij} = 0$ (Helmholtz theorem):
\begin{eqopt}[darkred]
    S^i_{,i} = 0 \qquad F^i_{,i} = 0 \qquad h^i_{j,i} =0 = h^i_i
\end{eqopt}
$\delta g_{\mu\nu}$ is symmetric thus has 10 d.o.f., indeed we have 4 scalar, 2 vector (with 2 d.o.f. each) and 1 tensor (with $3(3+1)/2 - 1 -3 = 2$ d.o.f.) perturbations: 10 functions specifying the perturbation. Not all are independent, though. 
The gauge freedom of the metric allows us to set some of them to zero\footnote{The gauge freedom of the metric is a consequence of the diffeomorphism invariance of GR. We can always choose a coordinate system in which the metric takes a certain form. This is called a gauge choice.}.

\begin{equation}
    x^\mu \rightarrow \tilde{x}^{\mu} = x^{\mu} + \xi^{\mu}(x) \qquad \text{gives} \qquad \tilde{g}_{\mu\nu}(\tilde{x}) = \frac{\partial x^\rho}{\partial\tilde{x}^\mu}\frac{\partial x^\sigma}{\partial\tilde{x}^\nu}g_{\rho\sigma}(x)
\end{equation}
From this and equation~\eqref{perturbation} you can see that the perturbation transforms as
\begin{equation}
    \delta \tilde{g}_{\mu\nu} = \delta g_{\mu\nu} - \bar{g}_{\mu\nu,\sigma} \xi^\sigma - \bar{g}_{\mu\sigma} \xi^\sigma_{,\nu} - \bar{g}_{\nu\rho} \xi^\rho_{,\mu} 
\end{equation}
Expand $\xi^\mu = \left(\xi^0,\xi^i_\perp+\xi^{,i}\right)$ using Helmholtz decomposition. Find $\delta \tilde{g}_{00}, \delta \tilde{g}_{0i}, \delta \tilde{g}_{ii} $.   


%%Tensor Perturbations
%\begin{mycolorbox}[purple!30!red]
%    \textbf{Tensor perturbations:}  
%
%    \begin{equation*}
%        ds^2=a^2\left\{d\tau^2 - \left[\delta_{ij}+h_{ij}\right]dx^i dx^j\right\}
%        \end{equation*}
%    
%    $\tilde{h}_{ij} = h_{ij} \; \rightarrow \;$ gauge invariance.  
%\end{mycolorbox} 