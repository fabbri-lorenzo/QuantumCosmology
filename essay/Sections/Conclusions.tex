We have seen that scalar tensor theories present various advantages in cosmology, such as the possibility to drive cosmic acceleration without employing a cosmological constant. They can be a valid extensions of General Relativity, provided they are constructed to avoid instabilities and to respect the observational bounds on coupling parameters. Furthermore, when comparing different scalar-tensor models, one shall be aware of the issue of Jordan and Einstein frame formulations, since careless frame choices can lead to misleading physical claims. 

Crucially, the need to satisfy known constraints entails the need for screening mechanisms. Among all, we discussed \textit{spontaneous scalarization}, which is not a generic feature of all modified gravity theories but arises in a well defined subclass whose curvature couplings trigger a tachyonic instability only above a critical threshold.  This sharp, self-regulating mechanism both protects low-curvature tests of gravity and endows compact objects with scalar “hair”, illustrating how modifications to General Relativity can remain hidden in the familiar regime yet reveal new dynamics where gravity grows strong.