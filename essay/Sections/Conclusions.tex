STTs are a generalization of the Brans-Dicke theory to allow the BD coupling to be a function of the scalar field~\cite{Quiros:2019ktw}: $\omega_{BD} \rightarrow \omega(\phi)$. The resulting action is
\begin{equation}
    S_{\text{ST}} = \int d^{4}x\,\sqrt{-g}\,\Bigl[\,\phi R - \frac{\omega(\phi)}{\phi}(\partial\phi)^{2} - 2V(\phi) + 2\mathcal{L}_{m}\Bigr]
\end{equation}
and it is simple to show that it is equivalent to 
\begin{equation}
S_{\text{ST}} = \int d^{4}x\,\sqrt{-g}\,\Bigl[f(\phi)\,R - \omega(\phi)(\partial\phi)^{2} - 2V(\phi) + 2\mathcal{L}_{m}\Bigr]
\end{equation}
by a redefinition of $f(\phi) \rightarrow \phi$ and $\omega(\phi) \rightarrow  \omega(\phi)f(\phi)/\bigl(\partial_{\phi}f\bigr)^{2}$.
STTs can also be further generalized to include higher-order derivatives of the scalar field and thus many other models can be constructed. 

As shown in Section~\ref{sec:expansion}, scalar-tensor frameworks are flexible enough to replace the cold-dark-matter fluid with either the scalar itself or a scalar-induced modification of gravity. 
However, in practical terms, the price is tight constraints from laboratory, Solar-System and cosmological observations. Therefore, every successful model must contain a screening or suppression mechanism in order to reduce to Einstein's GR in the low gravity regime. 

