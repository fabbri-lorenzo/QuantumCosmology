General Relativity (GR) -- the gravitational framework underlying the $\Lambda$CDM cosmological model -- remains one of the most elegant and thoroughly tested theories in modern science. More than a century after Einstein introduced his field equations, GR still offers the most accurate description of both cosmological dynamics and astrophysical phenomena. 

Despite these triumphs, there are good reasons to explore extensions or modifications of GR. The most important regarding cosmology is probably the unclear nature of dark matter, i.e.\ components that interact only gravitationally yet dominate the energy budget in $\Lambda$CDM. It is natural to ask whether altered gravitational laws on galactic or cosmological scales might account for this dark sector. 
Furthermore, deviations from GR in strong-gravity or quantum-gravity regimes could remove the Big-Bang singularity or those inside black holes \cite{Braglia:2021axy}.

An historical milestone in this direction came from Dirac’s proposal that fundamental constants may vary with time; a notion later formalized in the Brans–Dicke (BD) theory, where Newton’s constant becomes a dynamical quantity controlled by a time-dependent scalar field. BD theory is the prototype of modern scalar–tensor theories (STT) of gravity, which are the focus of this essay.

Chapter 1 introduces the basics of BD theory, along with a discussion clarifying the relations between Jordan and Einstein frames.

Chapter 2 concerns $f(R)$ theories, their equivalence with STTs and their ability to generate accelerated cosmic expansion without the need for exotic matter or dark energy. 
The essay concludes with few words on the current status of STTs and related open problems.

In the whole essay the metric signature is $(-,+,+,+)$. We use natural units with $c=1$ and $\hbar=1$. The Einstein gravitational constant is denoted by $G$.
