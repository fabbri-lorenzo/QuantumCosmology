In STTs, gravitational phenomena are partly due to the curvature of spacetime and partly due to the scalar field that sets the strength of the gravitational interactions at each point in spacetime. 

We make this statement more clear introducing the simplest STT, the Brans-Dicke theory (BD). The action of the BD theory is given by
\begin{equation}\label{eq:actionBD}
    S_{BD} = \int d^4x \sqrt{-g} \left( \phi R - \frac{\omega_{BD}}{\phi} g^{\mu\nu} \nabla_\mu \phi \nabla_\nu \phi - 2V(\phi) +2\mathcal{L}_{m}(\psi,\partial\psi,g_{\mu\nu})\right) 
\end{equation}
where $\phi$ is the BD scalar field, $V(\phi)$ is the self-interaction potential for $\phi$ and $\omega_{BD}$ is a free constant called the \textit{BD parameter}. It is convenient to define the Lagrangian density for the scalar field $\phi$: $\,\mathcal{L}_{\phi} \equiv - \frac{\omega_{BD}}{\phi} \left(\partial \phi \right)^2- 2V(\phi)$.

Let us compare $S_{BD}$ with the Einstein-Hilbert action complemented with a matter piece in the form of a self-interacting scalar field $\varphi$.
\begin{equation}
    S_{EH} = \int d^4x \sqrt{-g} \left( \frac{1}{8\pi G} R + 2\mathcal{L}_{\varphi}\right) 
\end{equation}
with $\mathcal{L}_{\varphi}= -\left(\partial \varphi\right)^2 -2V(\varphi)$. This is simply Einstein's GR with a matter piece in the form of a perfect fluid \cite{Quiros:2019ktw}.
We can see that in~\eqref{eq:actionBD} $\phi$ plays the role of the gravitational coupling constant, which is now dynamical:
\begin{equation}\label{eq:modulationEq}
    \phi(x)=\frac{1}{8\pi G(x)}
\end{equation}